\problemset{Комбинаторика -- упражнения с лекции 1}

\begin{enumerate}
  % 1
  \item
    Сравним $\Phi(100;2,5)$ и $\Phi(100;3,5)$:\\
    $\Phi(100;2,5) = 11$\\
    $\Phi(100;3,5) = 7$\\
    Очевидно $\Phi(100;2,5) > \Phi(100;3,5)$\\ \\
    При использовании аппарата производящих функций разложим $f(z)=\frac{1}{(1-z^2)(1-z^3)}$ и $h(z) = \frac{1}{(1-z^5)(1-z^3)}$, и сравним коэффициенты при $z^{100}$
    
  % 2  
  \item
    $\Phi(10; 1,2,3,5) = \Phi(5; 1,2,3,5) + \Phi(10; 1,2,3) = \Phi(0; 1,2,3,5) + \Phi(5; 1,2,3) + \Phi(7; 1,2,3) + \Phi(10; 1,2) = 1 + \Phi(2; 1,2,3) + \Phi(5; 1,2) + \Phi(4; 1,2,3) + \Phi(7; 1,2) + \Phi(8; 1,2) + \Phi(10; 1) = 2 + \Phi(2; 1,2) + \Phi(3; 1,2) + \Phi(5; 1) + \Phi(1; 1,2,3) + \Phi(4; 1,2) + \Phi(5; 1,2) + \Phi(7; 1) + \Phi(6; 1,2) + \Phi(8; 1) = 20$\\
    Раскладываем в ряд производящую функцию $f(z) = \frac{1}{(1-z)(1-z^2)(1-z^3)(1-z^5)}$, находим коэффициент перед $z^{10}$: 20.
  
  % 3  
  \item
    Поскольку гирю каждого веса можно брать или не брать, для гири веса k получаем производящую функцию $f(z) = 1 + z^k$. Для того, чтобы различать гири одно веса, но разных цветов, будем использовать в их производящих функциях разные переменные. Получим результирующую функцию $F(z) = (1+z)(1+w)(1+z^2)(1+z^5)(1+z^{10})(1+w^{10})(1+z^{20})(1+z^{50}) = z^{78} + wz^{77} + w^{10}z^{68} + w^{11}z^{67} + ...$ \\
    {\bf Ответ:} 4 способа.
  
  % 4
  \item
    В первом случае у нас имеется неограниченное количество ящиков, в первый ящик можно положить от 0 до 9 единиц, во второй -- от 0 до 9 десятков, в третий -- от 0 до 9 сотен, и т.д. Таким образом всего мы можем набрать любое количество предметов, т.е. производящей функцией будет $f(z) = 1 + z + z^2 + z^3 + ... = \frac{1}{1-z}$. (проводим аналогию с десятичной системой счисления)\\ \\
    Во втором случае у нас также имеется неограниченное количество ящиков, в первый можно положить 0 или 1 единицу, во второй -- 0 или 1 пару, в третий -- 0 или 1 четвёрку, и т.д. Таким образом мы также можем набрать любое количество предметов, т.е. производящей функцией будет $f(z) = 1 + z + z^2 + z^3 + ... = \frac{1}{1-z}$. (проводим аналогию с двоичной системой счисления)
\end{enumerate}

\begin{enumerate}
  %1
  \item
    $p_k(n) = p_{k-1}(n-1) + p_k(n-k)$\\
    Просуммируем по $k$ от 1 до $n$:\\
    $\sum\limits_{k=1}^{n} p_k(n)$
  
  %2
  \item
  
  %3
  \item
  
  %4
  \item
  
  %5
  \item
  
  %6
  \item
\end{enumerate}

\problemset{Комбинаторика -- упражнения с лекции 2}

\begin{enumerate}
  \item
    $S(n+1,k+1) = \sum\limits_{i=k}^{n}{n \choose i}S(i,k) = S(n,k) + \sum\limits_{i=k}^{n-1}{n \choose i} S(i,k)$ 
    
  \item
  

  \item

  
\end{enumerate}