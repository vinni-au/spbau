\problemset{Комбинаторика - упражнения с лекции 1}

\begin{enumerate}
  \item
    Сравним $\Phi(100;2,5)$ и $\Phi(100;3,5)$:\\
    $\Phi(100;2,5) = \Phi(98;2,5) + \Phi(100; 5)$\\
    $\Phi(100;3,5) = \Phi(97;3,5) + \Phi(100; 5)$\\
    Продолжая раскрытие по рекуррентной формуле, заметим, что для получения результата нам требуется сравнивать числа вида $\Phi(100-2k;2,5)$ и $\Phi(100-3k;3,5)$, и числа вида $\Phi(100-2k;5)$ и $\Phi(100-3k;5)$\\
    В силу свойств $\Phi$ делаем вывод, что $\Phi(100;2,5) > \Phi(100;5)$\\ \\
    При использовании аппарата производящих функции сравним $$\phi(z)=\frac{1}{(1-z^2)(1-z^5)} \text{ и } \psi(z)=\frac{1}{(1-z^3)(1-z^5)}$$\\
    $\phi(z) - \psi(z) = \frac{(1-z^2) - (1-z^3)}{(1-z^2)(1-z^3)(1-z^5)} = \frac{z^3-z^2}{(1-z^2)(1-z^3)(1-z^5)} > 0 \text{ при } |z| < 1$,
    значит $\phi(z) > \psi(z)$
    
  \item
    $\Phi(10; 1,2,3,5) = \Phi(5; 1,2,3,5) + \Phi(10; 1,2,3) = \Phi(0; 1,2,3,5) + \Phi(5; 1,2,3) + \Phi(7; 1,2,3) + \Phi(10; 1,2) = 1 + \Phi(2; 1,2,3) + \Phi(5; 1,2) + \Phi(4; 1,2,3) + \Phi(7; 1,2) + \Phi(8; 1,2) + \Phi(10; 1) = 2 + \Phi(2; 1,2) + \Phi(3; 1,2) + \Phi(5; 1) + \Phi(1; 1,2,3) + \Phi(4; 1,2) + \Phi(5; 1,2) + \Phi(7; 1) + \Phi(6; 1,2) + \Phi(8; 1) = 20$\\
    $f(z) = \frac{1}{(1-z)(1-z^2)(1-z^3)(1-z^5)}$
    
  \item
\end{enumerate}