\problemset{Задания 0(л)}

\begin{enumerate}
  % 1
  \item +    
  % 2  
  \item +  
  % 3  
  \item +  
  % 4
  \item +
\end{enumerate}


\begin{enumerate}
  %1
  \item +
  
  %2
  \item
  
	  \begin{enumerate}
	  %a
	  \item +	  
	  %b
	  \item +	  
	  %c
	  \item +	  
	  %d
	  \item +
	  
	  %e
	  \item

	  \end{enumerate}
  
  %3
  \item +
  
  %4
  \item
  
  %5
  \item
  
  %6
  \item
  
    \begin{enumerate}
    %a
    \item +
    
    %b
    \item
    
    %c
    \item
    \end{enumerate}
\end{enumerate}

\begin{enumerate}
  \item[5]
	  \begin{enumerate}
	  %a
	  \item
	  
	  %b
	  \item
	  
	  %c
	  \item
	  \end{enumerate}
\end{enumerate}

\problemset{Задания 1(л)}
\begin{enumerate}
  \item[6]
    \begin{enumerate}
      %1
      \item
      $ {n \choose k} $ -- количество способов выбрать $k$ элементов, которые остаются на своих местах, а $D_{n-k}$ -- количество способов переставить оставшиеcя элементы, не оставив ни один из них на месте. Таким образом:
      
      $$
D(n,k) = {n \choose k} D_{n-k} = \frac{n!}{(n-k)!k!} (n-k)! \left( 1 - \frac{1}{1!} + \frac{1}{2!} - ... + \frac{(-1)^{n-k}}{(n-k)!} \right) = \frac{n!}{k!} \sum\limits_{i=0}^{n-k} \frac{(-1)^i}{i!}
      $$
      
Тогда производящая функция будет следующая: $D(z,t) = F(z,t)D(z)$
$$F(z,t) = \sum\limits_{n=0}^{+\infty} \sum\limits_{k=0}^{n} {n \choose k} \frac{t^k z^n}{n!} = e^{(t+1)z}$$
$$ D(z) = \frac{e^{-z}}{1-z}
$$

Итак:

$$
D(z,t) = \frac{e^{tz}}{(1-z)}
$$
      %2
      \item +
      
      %3
      \item
    \end{enumerate}
\end{enumerate}

\problemset{Задания 2(л)}

\begin{enumerate}
  %1
  \item
    \begin{enumerate}
      %1
      \item +
      %2
      \item +
      %3
      \item +
      %4
      \item +
      
      %5
      \item
      
    \end{enumerate}
  
  %2
  \item
    \begin{enumerate}
      %1
      \item +
      %2
      \item +
      %3
      \item +
    \end{enumerate}
\end{enumerate}

\problemset{Задание 1(п)}
\begin{enumerate}
  %1
  \item +  
  
  %2
  \item
  
  %3
  \item +

  %4
  \item
\end{enumerate}

\problemset{Задание 2(п)}

\begin{enumerate}
  %1
  \item +
    
  %2
  \item
  
  %3
  \item
\end{enumerate}

\problemset{Задание 3(л)}

\begin{enumerate}
  %1
  \item + 
  %2
  \item +
  %3
  \item

  %4
  \item

  %5
  \item

\end{enumerate}