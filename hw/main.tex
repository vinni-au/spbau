% utf8
\documentclass{amsart}

\usepackage[utf8]{inputenc}
\usepackage{amsthm,amsmath,amsfonts,amssymb}
\usepackage[english,russian]{babel}
\usepackage{fullpage}
\usepackage{listings}
\usepackage{color}
\usepackage{xcolor}

\newtheorem{problem}{Задача}

\begin{document}

  \definecolor{dkgreen}{rgb}{0,0.6,0}
  \definecolor{gray}{rgb}{0.5,0.5,0.5}
  \definecolor{mauve}{rgb}{0.58,0,0.82}  

  \newcommand{\problemset}[1]{
    
    \begin{center}
      \Large #1
    \end{center}
  }

  \lstset{ %
    language=C++,                % the language of the code
    basicstyle=\footnotesize,           % the size of the fonts that are used for the code
    numbers=left,                   % where to put the line-numbers
    numberstyle=\tiny\color{gray},  % the style that is used for the line-numbers
    stepnumber=1,                   % the step between two line-numbers. If it's 1, each line 
                                    % will be numbered
    numbersep=5pt,                  % how far the line-numbers are from the code
    backgroundcolor=\color{white},      % choose the background color. You must add \usepackage{color}
    showspaces=false,               % show spaces adding particular underscores
    showstringspaces=false,         % underline spaces within strings
    showtabs=true,                 % show tabs within strings adding particular underscores
    frame=single,                   % adds a frame around the code
    rulecolor=\color{black!10},        % if not set, the frame-color may be changed on line-breaks within not-black text (e.g. comments (green here))
    tabsize=2,                      % sets default tabsize to 2 spaces
    captionpos=b,                   % sets the caption-position to bottom
    breaklines=true,                % sets automatic line breaking
    breakatwhitespace=false,        % sets if automatic breaks should only happen at whitespace
    title=\lstname,                   % show the filename of files included with \lstinputlisting;
                                    % also try caption instead of title
    keywordstyle=\color{blue},          % keyword style
    commentstyle=\color{dkgreen},       % comment style
    stringstyle=\color{mauve},        % string literal style
    escapeinside={\%*}{*)},            % if you want to add LaTeX within your code
    morekeywords={done, to},              % if you want to add more keywords to the set
  %  deletekeywords={...}              % if you want to delete keywords from the given language
  }

  \begin{tabbing}
\hspace{11cm} \= Студент: \= Сторожев Антон \\
  \> Группа: \> SE \\
%  \> Дата: \> \today
\end{tabbing}
\hrule
\vspace{1cm}


%  \problemset{Комбинаторика - упражнения с лекции 1}

\begin{enumerate}
  % 1
  \item
    Сравним $\Phi(100;2,5)$ и $\Phi(100;3,5)$:\\
    $\Phi(100;2,5) = \Phi(98;2,5) + \Phi(100; 5)$\\
    $\Phi(100;3,5) = \Phi(97;3,5) + \Phi(100; 5)$\\
    Продолжая раскрытие по рекуррентной формуле, заметим, что для получения результата нам требуется сравнивать числа вида $\Phi(100-2k;2,5)$ и $\Phi(100-3k;3,5)$, и числа вида $\Phi(100-2k;5)$ и $\Phi(100-3k;5)$\\
    В силу свойств $\Phi$ делаем вывод, что $\Phi(100;2,5) > \Phi(100;5)$\\ \\
    При использовании аппарата производящих функции сравним $$\phi(z)=\frac{1}{(1-z^2)(1-z^5)} \text{ и } \psi(z)=\frac{1}{(1-z^3)(1-z^5)}$$\\
    $\phi(z) - \psi(z) = \frac{(1-z^2) - (1-z^3)}{(1-z^2)(1-z^3)(1-z^5)} = \frac{z^3-z^2}{(1-z^2)(1-z^3)(1-z^5)} > 0 \text{ при } |z| < 1$,
    значит $\phi(z) > \psi(z)$
    
  % 2  
  \item
    $\Phi(10; 1,2,3,5) = \Phi(5; 1,2,3,5) + \Phi(10; 1,2,3) = \Phi(0; 1,2,3,5) + \Phi(5; 1,2,3) + \Phi(7; 1,2,3) + \Phi(10; 1,2) = 1 + \Phi(2; 1,2,3) + \Phi(5; 1,2) + \Phi(4; 1,2,3) + \Phi(7; 1,2) + \Phi(8; 1,2) + \Phi(10; 1) = 2 + \Phi(2; 1,2) + \Phi(3; 1,2) + \Phi(5; 1) + \Phi(1; 1,2,3) + \Phi(4; 1,2) + \Phi(5; 1,2) + \Phi(7; 1) + \Phi(6; 1,2) + \Phi(8; 1) = 20$\\
    $f(z) = \frac{1}{(1-z)(1-z^2)(1-z^3)(1-z^5)}$
  
  % 3  
  \item
    Сведём данную задачу к задаче раскладки неразличимых предметов по различимым ящикам. В нашем распоряжении имеется восемь ящиков, в два из них можно складывать любое количество предметов, ещё в два -- количество предметов, кратное 10, в оставшиеся -- количество предметов, кратное двум, пяти, 20ти и 50ти. В нашем распоряжении 78 предметов.\\
  
  % 4
  \item
    В первом случае у нас имеется неограниченное количество ящиков, в первый ящик можно положить от 0 до 9 единиц, во второй -- от 0 до 9 десятков, в третий -- от 0 до 9 сотен, и т.д. Таким образом всего мы можем набрать любое количество предметов, т.е. производящей функцией будет $f(z) = 1 + z + z^2 + z^3 + ... = \frac{1}{1-z}$. (проводим аналогию с десятичной системой счисления)\\ \\
    Во втором случае у нас также имеется неограниченное количество ящиков, в первый можно положить 0 или 1 единицу, во второй -- 0 или 1 пару, в третий -- 0 или 1 четвёрку, и т.д. Таким образом мы также можем набрать любое количество предметов, т.е. производящей функцией будет $f(z) = 1 + z + z^2 + z^3 + ... = \frac{1}{1-z}$. (проводим аналогию с двоичной системой счисления)
\end{enumerate}

\begin{enumerate}
  \item
  \item
  \item
\end{enumerate}

  \problemset {Домашнее задание 1}

\begin{enumerate}
  \item
    \begin{itemize}
      \item
        $(\lambda xyz.~zyx)yz(\lambda pq.~q) = (\lambda pq.~q)zy = y$
        
      \item
        $(\lambda yz.~zy)((\lambda x.~xxx)(\lambda x.~xxx))(\lambda y.~xxx) = (\lambda y.~xxx)((\lambda x.~xxx)(\lambda x.~xxx)) = xxx$
      
      \item 
       $SKSKSK = (\lambda xyz.~xz(yz))KSKSK = KK(SK)SK = (\lambda xy.~x)K(SK)SK = KSK = (\lambda xy.~x)SK = S$
      
    \end{itemize}    
    
  \item
    \begin{itemize}
      \item
        Введём вспомогательный предикат, для проверки, что число является 0:\\
        $iszero := \lambda x.~x (\lambda y.~false)~true$\\
        $ge := \lambda xy.~if~iszero (y~pred~x)~false~true = \lambda xy.~iszero~(y~pred~x)~false~true$
        
      \item
        $am := \lambda xy.~ if~(ge~x~y)~(y~pred~x)~(x~pred~y) = \lambda xy.~ ge~x~y~(y~pred~x)~(x~pred~y)$\\
      
    \end{itemize}
\end{enumerate}
\end{document}
